\documentclass[9pt,mathserif]{beamer}

%% \usepackage{pgfpages}
%% %\setbeamertemplate{note page}[plain]
%% %\setbeameroption{show notes on second screen=bottom}

\usepackage[spanish, es-noshorthands]{babel} 
\usepackage[utf8]{inputenc}

\usepackage{textpos} %textblock
%% \usepackage{tikz,pgfplots}
%% %\setbeamertemplate{navigation symbols}{}

%% %%%%\usecolortheme{rose}
%% %\usecolortheme{seahorse}
%% %\usetheme{Copenhagen}
%% %\usetheme{Warsaw}
\usetheme[width=3\baselineskip]{Berkeley}
\usecolortheme{spruce}   % color verde lindo
\definecolor{forestgreen}{rgb}{0.0, 0.27, 0.13}
\setbeamercolor{itemize item}{fg=forestgreen}
\setbeamercolor{block title}{bg=forestgreen!12,fg=forestgreen}
%% \usecolortheme{beaver}



\usepackage{subfigure}
\setbeamertemplate{caption}{\raggedright\insertcaption\par}
\setbeamerfont{caption}{size=\scriptsize}
\setlength{\abovecaptionskip}{5pt plus 3pt minus 2pt}

% TITLE PAGE ----------------------------------------------------------------------------------
%\logo{\includegraphics[width=0.75cm,height=0.75cm]{figuras/ib.png}}

\title[]{MODELADO COMPUTACIONAL DEL COMPORTAMIENTO HIDRODINÁMICO DE ELEMENTOS COMBUSTIBLES NUCLEARES}
\author[]{\large{Julia Martorana, Exequiel Fogliatto, Federico Teruel, Enzo Dari y Mariano Cantero}}
\institute{\normalsize{Departamento de Mecánica Computacional \\ Centro Atómico Bariloche \\  Comisión Nacional de Energía Atómica \\  Instituto Balseiro - Universidad Nacional de Cuyo}}
\date{}
% -----------------------------------------------------------------------------------------------

\begin{document}
\renewcommand{\tablename}{}                         % Reemplaza la palabra Cuadros, por 'nada' 
\renewcommand{\figurename}{}                        % Reemplaza la palabra Cuadros, por 'nada'

\begingroup
\makeatletter
\setlength{\hoffset}{-.5\beamer@sidebarwidth}
\makeatother
\begin{frame}[plain]
  \titlepage
  \begin{textblock}{5}(-0.5,-0.25)
    \includegraphics[width=1.15cm]{figuras/ENIEF2017.png}
  \end{textblock}
  \begin{textblock}{5}(12.25,-0.5)
    \includegraphics[width=1.15cm]{figuras/CNEA.jpg}
  \end{textblock}
  \centering
  \normalsize{XXIII Congreso sobre Métodos Numéricos y sus Aplicaciones} \\
\end{frame}
\endgroup

% -----------------------------------------------------------------------------------------------
% INDICE 
%% \frame[plain]{
%%   \frametitle{Contenidos}
%%   %	\begin{minipage}{\textwidth}
%%    \tableofcontents 
%%   %\end{minipage}
%% }

%% % -----------------------------------------------------------------------------------------------
 \section{Introducción}

\frame{
  \frametitle{Introducción}
  \framesubtitle{Elementos combustibles}

   \begin{minipage}{0.6\textwidth}
    \begin{figure}[!htb]
      \center
      \includegraphics[width=5.25cm]{figsINTRO/atucha2.jpg} \vspace{-2mm}
      \caption{Vista aérea de la Central Nuclear Atucha II.}
    \end{figure}
  \end{minipage}
   \begin{minipage}{0.35\textwidth}
     \vspace{-0.5cm}
    \begin{figure}[!htb]
      \center
      \includegraphics[width=3cm]{figsINTRO/EECC_atuchaI.jpg} \vspace{-2mm}
      \caption{EECC de la CN Atucha I.}
    \end{figure}
  \end{minipage}

  \vspace{-0.5cm}
   \begin{minipage}{0.6\textwidth}
    \begin{figure}[!htb]
    \center
    \includegraphics[width=4.75cm]{figsCAREM/CAREMfromCNEAfuelelement.jpg} \vspace{-2mm}
    \caption{Elemento combustible CAREM 25.} 
    \end{figure}
  \end{minipage}
  \begin{minipage}{0.35\textwidth}
    \begin{figure}[!htb]
      \center
      \includegraphics[width=3cm]{figsINTRO/carem.png} \vspace{-2mm}
      \caption{Esquema de funcionamiento del reactor CAREM.[1]}
    \end{figure}
  \end{minipage}
    \footnotetext[1]{\tiny{Laboratorio THI, CAB, CNEA}}
    
  
    \note{

      \begin{itemize}
      \item Una central nucleoeléctrica es básicamente un reactor nuclear que se utiliza para generar energía eléctrica. Las mismas producen vapor que posteriormente mueve una turbina acoplada a un generador eléctrico.
      \item El aumento de la produccion de energia que genera una central nuclear esta asociado con la capacidad de refrigeración de los componentes que conforman el núcle del reactor, en particular de los EECC
       \item Lo que limita la refrigeración de los EECC es el fenómeno conocido como Flujo critico de calor (CHF) que está relacionado con la generación y dinámica del vapor (ver algo más)
        
 
        \end{itemize}

    }
  
    %% \begin{itemize}
    %%   \item Se llevan a cabo cálculos numéricos como complemento de estudios de CNAII en un loop de freón.
    %%   \item Se desea evaluar el desempeño termohidráulico de separadores de CAREM.
    %%   \item El flujo es turbulento, se producen vórtices y se requiere la identificación de puntos calientes.
    %%   \item Fenómeno de flujo crítico de calor, optimización del quemado de EC.
    %% \end{itemize}

  

}

% =====================================================================================================
\frame{
  \frametitle{Introducción}
  \framesubtitle{Loop de Freón - Laboratorio de Termohidráulica - CAB - CNEA}

  \begin{minipage}{0.6\textwidth}
    \begin{figure}[!htb]
      \center
      \includegraphics[width=5.5cm]{figsINTRO/freon2b.png}
      \caption{Circuito de ensayos de CHF, Laboratorio THI, CAB.[1]}
    \end{figure}
  \end{minipage}
  \begin{minipage}{0.35\textwidth}
    \begin{figure}[!htb]
      \center
      \includegraphics[width=4cm]{figsINTRO/freon4.png}
      \caption{Reconstrucción del perfil térmico en un EC.[1]}
    \end{figure}
  \end{minipage}
  
  \begin{itemize}
  \item Circuito de ensayos para determinar el CHF en EECC.
  \item Posibilidad de ensayar los diferentes diseños de EECC de las centrales nucleares argentinas
  \item Permite llevar a cabo procesos de optimización: incorporar cambios geométricos para mejorar la transferencia térmica.
  \end{itemize}

  \footnotetext[1]{\tiny{Laboratorio THI, CAB, CNEA}}
  
  \note{\begin{itemize}
    \item El circuito de ensayo de CHF...
    \item En la figura se muestra...
      \end{itemize}
  }
  
}

% =====================================================================================================
\section{Objetivos}
\frame{
  \frametitle{Objetivos}

   \begin{block}{Objetivo general}
    Contribuir al desarrollo de la ingenería y optimización del diseño de elementos combustibles nucleares.
  \end{block}
  %\end{minipage}
  \vspace{1cm}
  
  %\begin{minipage}{0.75\textwidth}
  \begin{block}{Objetivos particulares}
  \begin{itemize}
    \item Realizar cálculos numéricos como complemento de estudios de CNAII en un loop de freón.
    \item Realizar cálculos numéricos para evaluar el desempeño termohidráulico de separadores de CAREM.
  \end{itemize}
  \end{block}
  %\end{minipage}

}

% =====================================================================================================
\section{Herramientas}
\frame{
  \frametitle{Herramientas}
  
  %\begin{minipage}{0.75\textwidth}
  \begin{block}{SALOME}
    Programa libre que incorpora módulos para generación de modelos CAD y motores de mallado en 3 dimensiones.
    \begin{itemize}
    \item Representación geométrica detallada de los EC.
    \end{itemize}
  \end{block}
  %\end{minipage}
  \vspace{1cm}
  
  %\begin{minipage}{0.75\textwidth}
  \begin{block}{OpenFOAM}
    Conjunto de bibliotecas de C++, destinadas a crear aplicaciones que involucren la resolución de EDP.
    \begin{itemize}
    \item Generación de mallas hexahédricas.
    \item Resolución de ecuaciones RANS mediante FVM.
    \end{itemize}
  \end{block}
  %\end{minipage}

    \note{-SnappyHexMesh es una utilidad de OpenFOAM que permite generar mallas hexahédricas para geometrías arbitrarias. Su desempeño es muy bueno.}

}


% =====================================================================================================
 \section{Geometría}
\frame{
  \frametitle{Geometría}
  \framesubtitle{Elemento combustible símil ATUCHA II}


  \begin{columns}
      \hspace{-1cm}
    \column{0.65\textwidth}
    \begin{minipage}[c][0.4\textheight][c]{\linewidth}
     \begin{figure}
       %\includegraphics[width=0.55\linewidth]{figsATUCHA/atucha_mesh1a.png}
     \includegraphics[width=0.525\linewidth]{figsATUCHA/2.png} \vspace{-3mm}
      \caption{Sección transversal del separador}
    \end{figure}
    \end{minipage}

    \vspace{0.5cm}
    \begin{minipage}[c][0.4\textheight][c]{\linewidth}
      \begin{figure}
        \includegraphics[width=0.65\linewidth]{figsATUCHA/atucha_mesh2a.png} \vspace{-3mm}
        \caption{Corte en el centro del separador.}
      \end{figure}
    \end{minipage}
    
    \column{0.5\textwidth}
    \begin{minipage}[c][0.4\textheight][c]{\linewidth}
      \begin{itemize}
      \item Símil CNAII 7 vainas.
      \item Malla de 14M de celdas.
      \item 4 capas adicionales en bordes.
      \end{itemize}
    \end{minipage}
    
    \begin{minipage}[c][0.5\textheight][c]{\linewidth}
      \begin{figure}
        \hspace{-1.5cm}   
        \includegraphics[width=0.95\linewidth]{figsATUCHA/atucha_mesh3a.png} \vspace{-3mm}
        \caption{Detalle de capas adicionales.}
        \end{figure}
    \end{minipage}
  \end{columns}
}

% =====================================================================================================
\frame{
  \frametitle{Geometría}
  \framesubtitle{Elemento combustible símil CAREM}
  
%  \vspace{-0.25cm}
  \begin{columns}
    \column{0.65\textwidth}
    \begin{minipage}[c][0.4\textheight][c]{\linewidth}
     \begin{figure}
        \includegraphics[width=0.85\linewidth]{figsCAREM/CAREM3.png} \vspace{-3mm}
        \caption{Geometría completa.}
    \end{figure}
    \end{minipage}

    %\vspace{0.5cm}
    \begin{minipage}[c][0.4\textheight][c]{\linewidth}

      \begin{itemize}
      \item Símil CAREM de 16 vainas y 3 tubos de control.
      \item Malla de 53M de celdas.
      \item 4 capas adicionales en bordes.
      \end{itemize}
      
    \end{minipage}
    
    \column{0.45\textwidth}
    \begin{minipage}[c][0.4\textheight][c]{\linewidth}
      \begin{figure}
     \includegraphics[width=0.8\linewidth]{figsCAREM/CAREM1.png} \vspace{-3mm}
      \caption{Vista de separador y resortes.}
      \end{figure}
    \end{minipage}
    
    \begin{minipage}[c][0.5\textheight][c]{\linewidth}
      \begin{figure}
        %\hspace{-1.5cm}   
        \includegraphics[width=0.75\linewidth]{figsCAREM/mesh_detalle3.png} \vspace{-3mm}
        \caption{Detalle de malla en resorte.}
        \end{figure}
    \end{minipage}
  \end{columns}
 
}

% ===========================================================================================================
\section{Resultados}
\subsection{Modelos de turbulencia}
\frame{
  \frametitle{Modelos de turbulencia}

    \begin{minipage}[c]{0.6\linewidth}
      \begin{itemize}
      \item Geometría: canal circular con 7 vainas.
      \item Análisis de modelos de turbulencia:
      \end{itemize}
    \end{minipage}
    \begin{minipage}[c]{0.3\linewidth}
      \begin{figure}
        \center
      \includegraphics[width=0.65\linewidth]{figsGEOM/geom_mod_turb.png} \vspace{-3mm}
      \caption{Sección transversal del canal analizado.}
    \end{figure}
    \end{minipage}
    
   \begin{table}[ht]
%      \setlength{\tabcolsep}{10pt}
        \centering
        \begin{tabular}{c c c c }
            \hline
            \bf Modelo & \bf Malla & \bf V. Layer  &  \bf Pendiente \\
            \hline
            \hline
            $k$ - $\omega$     & 1 / 0.25 & 0.25/10  &  4354\\ % 4.364
            $k$ - $\omega$ SST & 1 / 0.25 & 0.25/10  &  4819\\ %4.831
            Spallart Almaras & 0.125 / 0.125 & 0.25/10  &  4854 \\ %4.866
            \hline
            Mediciones & & & 5509 \\
            \hline
        \end{tabular}
        \caption{Modelos de turbulencia estudiados.}
    \end{table}    

}

\subsection{EC ATUCHA}
\frame{
  \frametitle{Elemento combustible símil ATUCHA}
  \framesubtitle{Pérdida de carga}
  \begin{figure}[!htb]
    \center
    \includegraphics[width=10cm]{figsATUCHA/US_DS.png}
  \end{figure}
}

\frame{
  \frametitle{Elemento combustible símil ATUCHA}
  \framesubtitle{Pérdida de carga}
  \begin{figure}[!htb]
    \center
    \includegraphics[width=8cm]{figsATUCHA/US_DS_fit_Q1.png}
  \end{figure}

  \begin{table}[ht]
        \centering
        \begin{tabular}{c | c | c | c }
            \bf Caudal $(m^3/s)$  & \bf $\Delta P_{sim}$  & \bf $\Delta P_{med}$ & \bf Diferencia (\%)  \\
            \hline
            \hline
            $Q_1=0.90 10^{-3}$ & 794.15  & 1246.43 & 36.3 (ver)\\
            $Q_2=1.72 10^{-3}$ & 2716.69 & 3910.64 & 30.5 (ver)\\
            $Q_3=2.01 10^{-3}$ & 3640.50 & 5220.35 & 30.3 (ver)\\
        \end{tabular}
        \caption{Valores de pérdida de carga local.}
        \label{tab:DP}
  \end{table}
  
}
\frame{
  \frametitle{Elemento combustible símil ATUCHA}
  \framesubtitle{Distribución de velocidad principal}
  \begin{figure}[ht]
        \centering
        %\begin{subfigure}[t]{0.45\textwidth}
            \centering
            \includegraphics[width=0.45\textwidth]{figsATUCHA/vel/U_0015US.png}
            \includegraphics[width=0.45\textwidth]{figsATUCHA/vel/U_00075US.png}
             
            \includegraphics[width=0.45\textwidth]{figsATUCHA/vel/U_half.png}
             \includegraphics[width=0.45\textwidth]{figsATUCHA/vel/U_00075DS.png}
             
            \includegraphics[width=0.45\textwidth]{figsATUCHA/vel/U_0015DS.png}
            \includegraphics[width=0.45\textwidth]{figsATUCHA/vel/U_003DS.png} \vspace{-3mm}
        \caption{Distribución de la velocidad principal en secciones transversales: a) a 15mm US del separador, b) a 7.5mm US del separador, c) en la mitad del separador, d) a 7.5mm DS del separador, e) a 15mm DS del separador y f) a 30mm DS del separador.}
    \end{figure}
 }
% ==========================================================================================================
\subsection{EC CAREM}
\frame{
  \frametitle{Elemento combustible símil CAREM}
  \framesubtitle{Pérdida de carga}
   \begin{figure}
      \includegraphics[width=10cm]{figsCAREM/US_DS.png} %\vspace{-3mm}
     % \caption{Sección transversal del separador}
   \end{figure}
}

\frame{
 \frametitle{Elemento combustible símil CAREM}
  \framesubtitle{Distribución de velocidades}
   \begin{figure}
       \includegraphics[width=10cm]{figsCAREM/CAREM4.png} %\vspace{-3mm}
     % \caption{Sección transversal del separador}
   \end{figure}
  
}

\frame{
 \frametitle{Elemento combustible símil CAREM}
  \framesubtitle{Distribución de velocidades}
   \begin{figure}
       \includegraphics[width=7cm]{figsCAREM/CAREM5.png} %\vspace{-3mm}
     % \caption{Sección transversal del separador}
   \end{figure}
  
}

\section{Conclusiones}
\frame{
  \frametitle{Conclusiones}

  \begin{itemize}
    \item Se llevaron a cabo simulaciones RANS en OpenFOAM.
    \item Las geometrías se mallaron con la herramienta SnappyHexMesh de OpenFOAM.
    \item Se realizó un análisis previo de varios modelos de turbulencia en un canal simple.
    \item Se calculó la pérdida de carga en un EC símil CNAII de 7 vainas y se comparó con mediciones experimentales. Se obtuvieron diferencias del 30\% (VER)
    \item Se realizó el cálculo de un EC símil CAREM de 16 vainas.
   \end{itemize}
}

%% \frame[plain]{
%%   \begin{textblock}{15}(4.5,1.25)
%%     \normalsize{¡Gracias por su atención!}
%%   \end{textblock}
%%   \begin{textblock}{15}(2.25,5)
%%     \normalsize{ XXIII Congreso sobre Métodos Numéricos y sus Aplicaciones}
%%   \end{textblock}
%%   \begin{textblock}{5}(0,4.25)
%%     \includegraphics[width=1.5cm]{figuras/ENIEF2017.png}
%%   \end{textblock}
  
%%   %% \begin{textblock}{5}(12,-4)
%%   %%   \includegraphics[width=1.25cm]{figuras/CNEA.jpg}
%%   %% \end{textblock}
%% }


\end{document}

